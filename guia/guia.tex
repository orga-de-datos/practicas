\documentclass{article}
\usepackage[table]{xcolor}
\usepackage{amsmath}
\usepackage{braket}
\usepackage{hyperref}
\usepackage[shortlabels]{enumitem}

\renewcommand{\emptyset}{\text{\O}}
\renewcommand{\thesubsection}{\thesection.\alph{subsection}}

\hypersetup{
    colorlinks=true,
    linkcolor=blue,
    filecolor=magenta,
    urlcolor=blue,
}

\title{Guia de Ejercicios}

\author{75.06 - Organización de Datos}

\date{v2020.08}

\begin{document}

\maketitle

\section{Ej 1}
Descargar el \href{https://gist.githubusercontent.com/CrossNox/6ea7fedb55bfb511449ba67d47d8e071/raw/3158919a26833fc8f2f3170b0e8a4d6f6cf09b8e/iris.csv}{iris dataset} como csv usando \texttt{pandas}.

\begin{enumerate}[a)]
\item Ver la cantidad de filas.
\item Ver los tipos de cada columna.
\item Ver los valores únicos para la columna \texttt{especie}
\item Calcular el promedio de cada columna (aquellas columnas donde tenga sentido).
\item Ver la cantidad de filas para cada valor único de la columna \texttt{especie}
\end{enumerate}
\end{document}
